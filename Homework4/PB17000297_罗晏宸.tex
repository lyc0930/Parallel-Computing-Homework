\documentclass{article}
\usepackage[UTF8]{ctex}
\usepackage[T1]{fontenc}
\usepackage[utf8]{inputenc}
\usepackage[colorlinks, linkcolor = black]{hyperref}
\usepackage{latexsym}
\usepackage{amsmath}
\usepackage{amssymb}
\usepackage{float}
\usepackage{xcolor}
\usepackage{geometry}
\usepackage{ulem}
\usepackage{listings}
\usepackage{siunitx}
 \lstset{
    basicstyle = \ttfamily,
    keywordstyle = \color{blue!80!black}\bfseries,
    linewidth = \linewidth,
    % xleftmargin=.1\textwidth,
    % xrightmargin=.1\textwidth,
    breaklines = true,
    breakatwhitespace = true,
    breakautoindent = true,
    frame = none,
    commentstyle = \small\color{gray},
    mathescape = true,
    escapeinside = ``,
    language = C++,
    moredelim=*[is][\uline]{!*}{!*},
    moredelim=[is][\color{blue!80!black}\bfseries\uline]{(*}{*)}
}
\geometry{left = 3.5cm, right = 3.5cm}
\title{Homework 4}
\author{PB17000297 罗晏宸}
\date{May 24 2020}

\begin{document}
\maketitle

\section{Exercise 15.1}
\begin{lstlisting}[xleftmargin = .1\textwidth, xrightmargin = .1\textwidth]
process 0:
MPI_Send(msg1, count1, MPI_INT, tag1, comm1);
parallel_fft($\cdots$);
process 1:
MPI_Send(msg1, count1, MPI_INT, tag1, comm1);
parallel_fft($\cdots$);
\end{lstlisting}
\subparagraph{\textcircled{1}} 试分析上述代码段的计算功能
\subparagraph{\textcircled{2}} 如果在 \lstinline{parallel_fft($\cdots$)} 中又包含了另一个发送程序:
\begin{lstlisting}[xleftmargin = .1\textwidth, xrightmargin = .1\textwidth]
    if (my_rank == 0) MPI_Send(msg2, count1, MPI_INT, 1, tag2, comm2);
\end{lstlisting}

如果没有通信体则会发生什么情况?

\paragraph{解}
\subparagraph{\textcircled{1}}
代码段中的 \lstinline{MPI_Send}函数参数列表中省略了 \lstinline{int dest} 参数,认为进程 \lstinline{0} 将起始地址为 \lstinline{msg1}的 \lstinline{count1} 个整型数据加上标签 \lstinline{tag1} 发送给通信域 \lstinline{comm1} 中的目标进程 \lstinline{1},进程 \lstinline{1} 将相同类型的消息缓冲发送给同一通信域中的目标进程 \lstinline{0} 。考虑到 \lstinline{parallel_fft($\cdots$)} 中的消息接收,代码通过两个进程实现了快速傅里叶变换的并行计算。

\subparagraph{\textcircled{2}}
没有通信体则程序无法区分在通信域 \lstinline{comm1} 和 \lstinline{comm2} 中编号同为 \lstinline{1} 的两个进程,发送给通信域 \lstinline{comm1} 中的目标进程 \lstinline{1} 的消息 \lstinline{msg1} 会与 \lstinline{msg2} 相互干扰,无法安全地区别不同的通信。
\section{Exercise 15.3}
填写空白处,使下述两代码段完全等效:
\subparagraph{\textcircled{1}}
\begin{lstlisting}[xleftmargin = .1\textwidth, xrightmargin = .1\textwidth, ]
float data[1024];
MPI_Datatype floattype;
MPI_Type_vector(10, 1, 32, MPI_FLOAT, &floattype);
MPI_Type_commit(&floattype);
MPI_Send(data, 1, floattype, dest, tag, MPI_COMM_WORLD);
MPI_Type_free(&floattype);
\end{lstlisting}
\subparagraph{\textcircled{2}}
\begin{lstlisting}[xleftmargin = .1\textwidth, xrightmargin = .1\textwidth]
float data[1024], buff[10];
for(`\uline{~~~~}`; `\uline{~~~~}`; i++) buff[i] = data[`\uline{~~~~}`];
MPI_Send(buff, `\uline{~~~~}`, MPI_FLOAT, `\uline{~~~~}`, `\uline{~~~~}`, `\uline{~~~~}`);
\end{lstlisting}


\paragraph{解}
\subparagraph{\textcircled{2}}
\begin{lstlisting}[xleftmargin = .1\textwidth, xrightmargin = .1\textwidth]
float data[1024], buff[10];
for(!*(*int*) i = 0!*; !*i < 10!*; i++)
    buff[i] = data[!*32 * i!*];
MPI_Send(buff, !*10!*, MPI_FLOAT, !*dest!*, !*tag!*, !*MPI_COMM_WORLD!*);
\end{lstlisting}

\section{Exercise 15.13}
( Buffon-Laplace 针问题)设想一个长为 $l$ 的针掉在一个等距平行线网格上,每个格的长和宽分别是 $a$ 和 $b$ 。针至少落在一根线上的概率为
$$
    P(l,\,a,\,b) = \frac{2l(a + b) - l^2}{\pi a b}
$$
我们可以用蒙特卡洛模拟法进行投针,从而来估计$\pi$的值
\subparagraph{\textcircled{1}} 用 C 语言写一个串行的 Buffon-Laplace 针问题的仿真程序。程序打印 $\pi$ 的值。当针的数量是一百万时,运行模拟的时间是什么? $\pi$ 的位数是多少?

\paragraph{解}
\subparagraph{\textcircled{1}}
详细程序可见附件,程序中使用了如下函数模拟统计投针落在线上的次数
\begin{lstlisting}[xleftmargin = .1\textwidth, xrightmargin = .1\textwidth, frame = single]
int BuffonLaplaceSimulation(double l, double a, double b, int n)
{
    int hits = 0;
    double x1, y1, x2, y2;
    double angle;
    for (int i = 0; i < n; i++)
    {
        x1 = a * (double)rand() / (double)RAND_MAX;
        y1 = b * (double)rand() / (double)RAND_MAX;
        angle = 2.0 * M_PI * (double)rand() / (double)RAND_MAX;
        x2 = x1 + l * cos(angle);
        y2 = y1 + l * sin(angle);
        if (x2 <= 0 || x2 >= a || y2 <= 0 || y2 >= b)
            hits++;
    }
    return hits;
}
\end{lstlisting}
当取$l = a = b = 2$时,进行一百万次投针
程序运行时间约为\SI{0.183}{\second},计算得到$\pi$结果精确到小数点后 2 位,更改$l,\,a,\,b$的值,程序运行时间没有显著变化,结果精确位数没有改变。
\end{document}