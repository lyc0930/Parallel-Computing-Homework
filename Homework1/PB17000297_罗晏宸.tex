\documentclass{article}
\usepackage[UTF8]{ctex}
\usepackage[T1]{fontenc}
\usepackage[utf8]{inputenc}
\usepackage{latexsym}
\usepackage{amsmath}
\usepackage{siunitx}
\usepackage{float}
\title{Homework 1}
\author{PB17000297 罗晏宸}
\date{March 1 2020}

\begin{document}
\maketitle

\section{Exercise 1.11}
查阅资料,找出一个并行计算的典型应用,详细描述该应用在并行化方面成功和失败之处以及遇到的困难。试从下列方面考虑:
\subparagraph{1}该应用时针对什么学科或者工程上的具体问题设计的?
\subparagraph{2}对于要解决的问题,该应用实际效果怎么样?模拟结果和物理结果进行比较的结果如何?
\subparagraph{3}该应用的运行在什么并行计算平台上(如分布式、共享内存或向量机)?这个应用是使用哪种开发工具开发的?
\subparagraph{4}和所运行平台最佳性能相比较,该应用的实际工作性能怎样?
\subparagraph{5}该应用的可扩展性如何?如果不好,你认为它的扩展性的瓶颈在何处?

\paragraph{解}
下面是关于并行技术在集合数值天气预报系统中应用的描述。大气科学尤其数值模式的发展进步与并行计算息息相关,这一点已经成为气象预报和并行计算两个领域的共识。随着并行计算技术的不断发展进步,并行计算机的规模愈来愈大,计算能力愈来愈强,这使得更为快捷、准确地进行数值预报成为可能。
\par
集合数值天气预报应用运行在大规模并行处理机上,结合集合数值天气预报业务需求中大规模、高分辨率的特点,联合采用数据并行和消息传递并行编程模型。应用在计算机软件实施上遇到的主要困难是:样本数多、解算量和数据传输量都很大\cite{ref1}。在数值天气预报系统中,并行计算能够实现比较高的正确率,模拟结果和物理结果吻合度较高,但是预报产品的生成效率难以保证。
\par
对于集合数值天气预报而言,系统的解耦合性较好,通过修改产品生成的算法,可以适应工业、农业、生活降水预报等不同的数据预报需求,因此可扩展性是较好的。

\section{Supplementary Exercise}
根据调研的应用需求,预算150万人民币购置计算资源。请确定购买的机器配置。

\paragraph{解}
考虑到集合数值天气预报系统的应用需求以及其商品化的结果输出,较成熟的计算资源购置方案是购买超级计算机机时,参考中国国家网格相关作业的资源数,拟每月购买超算机时约$\SI{256}{\text{核}} \times \SI{10}{\text{天}} \approx \SI{61440}{\text{CPU小时}}$,结合$0.1 \sim \SI{0.2}{\text{核}\times\text{小时}}$的合约参考价,年投入约为\SI{110592}{\text{元}}。

\begin{thebibliography}{99}
    \bibitem{ref1}Zhang Yi. PARALLEL ALGORITHM IN SW ENSEMBLE NUMERICAL WEATHER PREDICTION SYSTEM[J]. Journal of Applied Meteorological Science, 2002, 13(2): 232-238
    \bibitem{ref2}中国国家网格 www.cngrid.org
\end{thebibliography}
\end{document}